\chapter{Conclusion}\label{chapter:Conclusion}

In conclusion, we assessed and compared different VECSEL chips based on their gain saturation at different temperatures. Specifically interesting was the comparison of a new VECSEL design incorporating a DBR for the pump wavelength. We analysed the saturation fluence, small signal reflectivity, and rollover parameter to gain provided insights into the performance and behavior of the VECSELs under various conditions.

Regarding the saturation fluence, we observed that independently of the conditions, it  stayed constant after population inversion was achieved. Further we saw an increase of the saturation fluence for lower temperatures. This could be explained by the thermal expansion/contraction of the gain structure and therefore shifting the resonance of the structure away/towards the probing laser wavelength. A comparison of the different structures showed no significant differences between the new and old VECSEL designs. However, for comparing the same structure but with different heat spreader, we could verify that the improved thermal conductivity of the diamond has indeed a bigger effect on the thermal properties of the active region.

The small signal reflectivity showed a linear growth followed by a saturation for higher pump powers. The saturation generally occurred at higher pump powers for colder temperatures. The comparison of the saturation behavior showed no significant difference between the no pump DBR and pump DBR VECSELs. The absolute values of the small signal reflectivities were a lot higher for the VECSEL chip with pump DBR than for the VECSEL without the pump DBR with the same amount of gain quantum wells. This means that the pumping of the pump DBR VECSEL is more efficient and also causes less heating of the structure.

The rollover parameter analysis showed a linear decrease versus pump powers. Which, considering that the pump heats up the active region, aligns with the findings of , that the two-photon absorption coefficient increases with temperature \cite{GaAsTemp}. When comparing the rollover parameter for the different structures, it was found that the VECSEL with pump DBR had approximately half the value compared to the no pump DBR VECSEL. This significant difference suggests the need for future investigation into the behaviour of the rollover parameter, with the aim of potentially resolving the issue with modelocking.

Comparing the results to previous work, the no pump DBR VECSEL showed consistency in the fit parameters and demonstrated a similar constant behavior between saturation fluence and pump power as reported in \cite{Gaulke2021HighCharacterization}. Additionally, the new structure incorporating a pump DBR and diamond heat spreader showed a record small signal gain of \qty{6.5}{\percent}, surpassing previous reported values.

Furthermore, we enhanced the measurement setup by automating the control of the pump laser diode. This improved the time efficiency of the measurements by 5 \% as well as allowing to fully autonomous measuring of data over arbitrary number of pump powers. For this the software interface was updated to include a list of current values for the power supply of the laser diode. The signal processing algorithm was also improved,  by using a lower triggering value and examining neighboring points and signal trends, the new method avoided triggering on noisy signal levels and ensured triggering on the total peak. This allowed for a successful automated measuring process. 