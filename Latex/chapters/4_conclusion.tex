\chapter{Conclusion}\label{chapter:Conclusion}

In conclusion, we assest and compared different VECSEL chips based on their gain saturation at different temperatures. 


sstructures involved using fitting parameters obtained from the model described in Equation (eq:model). To account for parameter correlations, we initially determined the non-saturable reflectivity by fitting the measurement with no pump power and then fixed the parameter for subsequent fits. This approach ensured that the fitting process accurately captured the non-saturable reflectivity originating from impurities in the structure, which remain unchanged during measurements. Additionally, we restricted the fitting range for high pulse fluences due to the complex and less understood nonlinear behavior observed in the reflectivity at high fluence levels.

In the subsequent sections, we examined each of the three remaining fitting parameters. We discussed the general trends and features of the measurements based on the results for the copper heat spreader with pump DBR (cDBR) VECSEL design and subsequently explored these features for the other structures.

Regarding the saturation fluence (Fsat), we analyzed its behavior for the cDBR VECSEL at different heatsink temperatures. We observed that the saturation fluence remained relatively low initially at zero pump power and then reached a constant value once population inversion was achieved within the gain region. However, before reaching the constant plateau, there was often a jump in the observed saturation fluence, which could be attributed to an artifact from the fitting process. The flat reflectivity curve at low pump powers made it challenging for the fitting method to accurately determine the saturation fluence.

Regarding temperature dependence, we found that colder temperatures resulted in higher saturation fluence, which might appear counterintuitive considering the decrease in carrier density at lower temperatures. However, the explanation lies in the design of the VECSEL. The optical thicknesses in the structure are designed for a specific wavelength, but thermal expansion can alter these thicknesses, consequently affecting the specific wavelength of the design. As the measurements were performed for a fixed wavelength, temperature variations influenced the resonance condition and, thus, the efficiency of carrier promotion, leading to observed changes in saturation fluence. This nonlinear behavior was also evident in the data.

Comparing the mean values of saturation fluence for the different structures and temperatures (see Table tab:fsat), we found no significant difference between the no pump DBR VECSEL and the cDBR VECSEL, which is reasonable considering that both chips have the same active region.

In the case of VECSELs with pump DBRs, comparing the cDBR VECSEL to the diamond heat spreader (dDBR) VECSEL, we observed that at \qty{10}{\celsius}, they exhibited similar saturation fluence. However, at colder heatsink temperatures, the dDBR VECSEL showed a higher saturation fluence. This can be attributed to the improved thermal conductivity of the diamond heat spreader, leading to more efficient cooling of the structure. As a result, thermal expansion becomes more significant, shifting the resonance of the structure closer to the probing wavelength and thereby increasing the saturation fluence.

The hybrid VECSEL yielded different results compared to the other structures. Cooling the structure in this case moved the resonance of the design further away from the probing wavelength, resulting in a decrease in saturation fluence.

In summary, our analysis of saturation fluence revealed temperature-dependent behavior and highlighted the influence of different heat spreader materials on the saturation fluence of the VECSEL structures. The results underscored the importance of considering thermal effects and their impact on resonance conditions and overall device performance.

Conclusion:

In this chapter, we assessed and compared different VECSEL structures using fitting parameters obtained from our model. We adopted an approach that accounted for parameter correlations and considered the non-saturable reflectivity by fixing the parameter obtained from fitting the measurement with no pump power. We also restricted the fitting range for high pulse fluences due to the complex and less understood behavior of the system at high fluence levels.

Firstly, we discussed the saturation fluence ($F_{sat}$) parameter. We observed that the saturation fluence remained constant once population inversion was reached, but there was often a jump in the observed saturation fluence before reaching the constant plateau. We attributed this jump to an artifact from the fitting process. Additionally, we found that the saturation fluence increased with colder temperatures, which may seem counterintuitive considering the decrease in carrier density. However, this behavior can be explained by the thermal expansion-induced shift in resonance, affecting the efficiency of carrier promotion.

Next, we analyzed the small signal reflectivity ($R_{ss}$) parameter. We observed a linear growth in the small signal reflectivity initially, followed by a drop-off at higher pulse fluences. This break-off point occurred at higher fluence values for colder temperatures. We noted that comparing the maximum reflectivity values across different VECSEL chips lacked significance due to spatial variation in the measurement spot. Instead, we focused on characterizing the break-off behavior and found no significant difference between the no pump DBR and pump DBR chips. Surprisingly, the break-off happened sooner for the diamond heat spreader compared to the copper heat spreader.

Finally, we examined the rollover parameter ($F_2$) and observed a linear decrease in its value for higher pump powers. This decrease indicated a higher likelihood of two-photon absorption in the active region. We also found that the rollover parameter had a higher value for colder temperatures, consistent with the temperature dependence of two-photon absorption.

Comparing our results with previous work, we could only make comparisons for the no pump DBR VECSEL. The fit parameters indicated consistency between our study and the previous work, and we observed the same constant behavior between saturation fluence and pump power as reported in the paper.

Overall, our investigation provided insights into the performance and characteristics of different VECSEL structures. We observed the influence of temperature, thermal effects, and material properties on the fitting parameters. Furthermore, our results highlighted the improved performance of the new structure incorporating a pump DBR and diamond heat spreader, achieving a record small signal gain. These findings contribute to the understanding and potential advancement of VECSEL technology.
